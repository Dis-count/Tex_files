\documentclass[UTF8]{article}
    \author {Discount}
    \title {Code Insert}
\usepackage{ctex}
\usepackage{amsmath}
\usepackage{amssymb}
\usepackage{color}
\usepackage{xcolor}
\usepackage{listings}
\usepackage{fontspec}

\lstset{
    basicstyle = \sffamily,   % 基本代码风格
    keywordstyle = \bfseries, % 关键字风格
    commentstyle = \rmfamily\itshape,  % 注释的风格,斜体
    stringstyle = \ttfamily, % 字符串风格
    flexiblecolumns,  % 别问为什么,加上这个
    numbers = left,   % 行号的位置在左边
    numbersep = 15pt,
    showspaces = false,  % 是否显示空格,显示了有点乱,所以不现实了
    numberstyle = \zihao{-5}\ttfamily\color{gray},    % 行号的样式,小五号,tt等宽字体
    showstringspaces = false,
    captionpos = t, % 这段代码的名字所呈现的位置,t指的是top上面
    frame=shadowbox,
    % frame = lrtb,   % 显示边框
    rulesepcolor=\color{red!20!green!20!blue!20},%代码块边框为淡青色
    extendedchars = false,
    xleftmargin = 3em,
    aboveskip = 2em,
    framexleftmargin = 4em,
    % title=\lstname,
}

\lstdefinestyle{Matlab}{
    language = Matlab, % 语言选 Matlab
    basicstyle = \zihao{-5}\ttfamily,
    numberstyle = \zihao{-5}\ttfamily,
    keywordstyle = \color{blue}\bfseries,
    keywordstyle = [2] \color{teal},
    stringstyle = \color{magenta},
    commentstyle = \color{red}\ttfamily,
    breaklines = true,   % 自动换行,建议不要写太长的行
    columns = fixed,  % 如果不加这一句,字间距就不固定,很丑,必须加
    basewidth = 0.5em,
}

\begin{document}


\lstinputlisting[
    style   =   Matlab,
    caption =  {\bf Construct.m},
    label   =  {Construct}
]{Code/Construct.m}

\lstinputlisting[
    style   =   Matlab,
    caption =  {\bf Pretreatment.m},
    label   =  {Pretreatment}
]{Code/Pretreatment.m}

\lstinputlisting[
    style   =   Matlab,
    caption =  {\bf Pm.m},
    label   =  {Pm}
]{Code/Pm.m}

\lstinputlisting[
    style   =   Matlab,
    caption =  {\bf TotalCost.m},
    label   =  {TotalCost}
]{Code/TotalCost.m}

\lstinputlisting[
    style   =   Matlab,
    caption =  {\bf IPC.m},
    label   =  {IPC}
]{Code/IPC.m}

\lstinputlisting[
    style   =   Matlab,
    caption =  {\bf CP.m},
    label   =  {CP}
]{Code/CP.m}

\lstinputlisting[
    style   =   Matlab,
    caption =  {\bf DP.m},
    label   =  {DP}
]{Code/DP.m}

\lstinputlisting[
    style   =   Matlab,
    caption =  {\bf Coalition.m},
    label   =  {Coalition}
]{Code/Coalition.m}

\lstinputlisting[
    style   =   Matlab,
    caption =  {\bf LP1.m},
    label   =  {LP1}
]{Code/LP1.m}

\lstinputlisting[
    style   =   Matlab,
    caption =  {\bf LP2.m},
    label   =  {LP2}
]{Code/LP2.m}

\lstinputlisting[
    style   =   Matlab,
    caption =  {\bf Players.m},
    label   =  {Players}
]{Code/Players.m}

\end{document}
