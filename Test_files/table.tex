\documentclass[UTF8]{article}

\usepackage{ctex}
\usepackage{amsmath}
\usepackage{amsthm}

\usepackage{booktabs}  % 用于三线表宏包
\usepackage{multirow}  % 多行合并表格

\usepackage{geometry}
\geometry{a4paper,scale=0.8}
\usepackage{graphicx}
\usepackage{amssymb}

\usepackage{setspace}
\renewcommand{\baselinestretch}{1.5}

\begin{document}

\newcommand{\tabincell}[2]{\begin{tabular}{@{}#1@{}}#2\end{tabular}}

\begin{table}[ht]
\tabcolsep=50pt
\small\renewcommand\arraystretch{2}

\caption{Different conditions used in this paper.\label{tab:1}}

{\begin{tabular}{lc}
\hline
Categories & Condition \\
\hline
Condition A & $nonincreasing,~\int_{\underline\beta}^{\overline\beta}f(x)^{2}dx> f(\overline{\beta})$ \\
\hline
Condition B & $\Bigg(f(x)\Big/\Big(\big(1-F(x)\big)\cdot F(x)\Big) \Bigg)^\prime\le 0$ \\
\hline
Condition C& \tabincell{c}{$\Big(log\big(f(x)\big)\Big)^{\prime\prime}\le 0$,\\$lim_{x \to \overline\beta}\Big[\big(1-F(x)\big)/f(x)\Big] <\infty$, $lim_{x \to \underline\beta}\big[F(x)/f(x)\big] <\infty$}  \\
\hline
\end{tabular}}
{}
\end{table}

\begin{tabular}{|c|r|r|}
	\hline
	% \multirow{2}{*}{Name},2为所占的行数,此语句可以使得内容垂直居中
	% \multicolumn{2}{c|}{Flag},2为所占的列数,格式由第二个{}控制
	% \cline{2-3}指本行的2,3列画横线
	\multirow{2}{*}{Name} & \multicolumn{2}{c|}{Flag}  \\ \cline{2-3}
	                      &  Yes  &   NO \\  \hline
	Index                 & 87    &  100 \\  \hline
\end{tabular}

\begin{table}[htbp]
	\centering  % 显示位置为中间
	\caption{standard table}  % 表格标题
	\label{table1}  % 用于索引表格的标签
	%字母的个数对应列数,|代表分割线
	% l代表左对齐,c代表居中,r代表右对齐
	\begin{tabular}{|c|c|c|c|}
		\hline  % 表格的横线
		1&2&3&4 \\  % 表格中的内容,用&分开,\\表示下一行
		\hline
		0.1&0.2&0.3&0.4 \\
		\hline
	\end{tabular}
\end{table}

\begin{table}[htbp]
	\centering
	\caption{three-line table}
	\begin{tabular}{cccc}
		\toprule  % 顶部线
		1&2&3&4 \\
		\midrule  % 中部线
		0.1&0.2&0.3&0.4 \\
		\bottomrule  % 底部线
	\end{tabular}
\end{table}

\end{document}
